\documentclass[11pt]{article}\usepackage[]{graphicx}\usepackage[]{color}
\usepackage{rotating}

\oddsidemargin=.25in
\evensidemargin=.25in
\textwidth=6in
\topmargin=0in
\textheight=9in

\parindent=0in
\pagestyle{empty}

\title{GIS enabled Hospital Readmissions Dashboard:
 Going beyond EMR – Patients "place" becomes clinically relevant\\
\smallskip
  \large RESOURCE REQUEST/IMPROVEMENT PLAN\\ }
\author{James Martinez}


\begin{document}

\maketitle
\newpage	
\tableofcontents
\newpage	


operations caused it to increase.

\section{HANDLING INSTRUCTIONS}

The title of this document is GIS enabled Hospital Readmissions Dashboard: Going beyond EMR – Patients "place" becomes clinically relevant – Resource Request/Improvement Plan.
\paragraph{}

The information gathered in this Resource Request/Improvement Plan is classified as ‘For Official Use Only’ and should be handled as sensitive information not to be disclosed. This document should be safeguarded, handled, transmitted, and stored in accordance with appropriate security directives.
\paragraph{}
Points of Contact
\paragraph{}


% latex table generated in R 3.1.3 by xtable 1.7-4 package
% Tue Apr 28 14:27:35 2015
\begin{table}[ht]
\centering
\begin{tabular}{rll}
  \hline
 & Local Level Contact&  \\ 
  \hline
1 &  &  \\ 
  2 & Name:  & Dora Barilla, Dr.PH. \\ 
  3 & Title:  & Principal Investigator \\ 
  4 & Center for Strategy and Innovation &  \\ 
  5 & Agency:  & Loma Linda University \\ 
  6 & Street Address:  & 11175 Mountain View Ave, Suite M \\ 
  7 & City, State ZIP:  & Loma Linda, CA, 92354 \\ 
  8 & Office:  & 909-558-3261 \\ 
  9 & email:  & dbarilla@llu.edu \\ 
   \hline
  10 &  &  \\ 
  11 & Name:  & Matt Keeling, B.S. \\ 
  12 & Title:  & GIS Solutions/ Data Analyst \\ 
  13 &                         Claremont Graduate University &  \\ 
  14 & Agency:  & School of Information Systems and Technology \\ 
  15 & Street Address:  & 130 East Ninth Street \\ 
  16 & City, State ZIP:  & Claremont, CA 91711-6190 \\ 
  17 & Office:  & 909-588-7941 \\ 
  18 & email: & john.keeling@cgu.edu \\ 
   \hline
  19 &  &  \\ 
  20 & Name:  & James Martinez, Ed.D. \\ 
  21 & Title:  & GIS Solutions / Data Analyst  \\ 
  22 &                         Claremont Graduate University &  \\ 
  23 & Agency:  & School of Information Systems and Technology \\ 
  24 & Street Address:  & 130 East Ninth Street \\ 
  25 & City, State ZIP:  & Claremont, CA 91711-6190 \\ 
  26 & Office:  & 909-588-3630 \\ 
  27 & email: & james.martinez@cgu.edu \\ 
   \hline
\end{tabular}
\end{table}

\newpage

\section{EXECUTIVE SUMMARY}

The improvement plan for a Geographic Information System (GIS) enabled hospital readmissions dashboard will assist hospital administrators, nurses and executives and hospital quality improvement efforts, and coordinating care for complex patients with conditions that represent 20\% of all readmissions. The hospital readmissions reduction program was established as part of the Affordable Care Act section 3025 and Social Security Act 1886(q) (cms.gov). CMS is penalizing hospitals with high rates of patient readmissions within 30 days of discharge for the following medical conditions: acute myocardial infarction, heart failure and pneumonia and recently added chronic lung problems and elective hip and knee surgery to the list. (cms.gov). This created a response of urgency by hospital administrators to create dynamic data platforms and medical response directions to help improve quality indicators and performance metrics for hospitals level of productivity and evidence-based approaches to comply with readmission policies and avoidable reimbursement penalties.
\paragraph{}
The significance of the hospital readmissions improvement plan is aligned with the operational values of the agency and supports the overall mission and compliance mandates. The GIS enabled platform also gathers performance-based measures to support hospitals and complex reporting demands and time sensitive decision making processes necessary to intervene with complex patients. The absence of more complex GIS capabilities limits hospitals and in particular administrators, nurses and executives from making more informed decisions. 
\paragraph{}
Additionally, the lack of more powerful data analysis platforms to hospitals identify, monitor and measure quality processes over time and translation into meaningful information that will collectively and collaboratively help improve quality indicators and performance metrics warrants the need for a GIS resource and improvement plan. Moreover, the inability to perform analysis on complex data would limit the type of performance-based measures hospitals could realistically provide communities without considering patients "place" as a clinically relevant indicator.
\paragraph{}
The improvement plan focuses on four target capabilities that will help guide the development of a fully integrated, cost effective, person centered hospital readmissions reduction dashboard that helps ensures that all patient needs are met before they are released or return to the hospital for the same condition within 30-days of discharge as defined by the Centers for Medicare and Medicaid Services (CMS).
\paragraph{}

\begin{itemize}

\item Planning: Hospital Readmissions Reduction - Strategic Plan 
\item Evaluating the Hospital Readmissions Reduction -  Business Needs 
\item Needs Assessment: Hospital Readmissions Reduction - Applications and Info Gathering 
\item Implementation Plan: Hospital Readmissions Reduction - Recognition of Indicators and Warning Capabilities

\end{itemize}
\paragraph{}
The platform development phase was composed of three members from LLU and ESRI, and feedback from 10 CGU students. The improvement plan team had various backgrounds and included hospital executives, planners, research strategists, data architects and GIS specialists.
\paragraph{}

The objectives in this improvement plan for a hospital readmissions dashboard are:

\begin{enumerate}
  \item Objective 1: Design a hospital readmission dashboard that can integrate with other
assessment platforms to enhance the identification of high-risk patients for avoidable 30-90 day readmissions after discharge (GIS needs analysis)
  \item	Objective 2: Allocation of resources to fulfill resource request, including routing for 	
nurses or care teams to follow-up on complex patients within 48 hours after discharge, home visits for collecting vitals, symptoms and medication education and prescription refills if needed (components)
  \item	Objective 3: Evaluate robustness of improvement plan and response for each condition category groups MI, HF, PN, COPD and THA/TKA (requirements)
  \item	Objective 4: Planned Implementation of the GIS resource and improvement plan
  \item	Objective 5: Value regarding the improvement plan (costs and benefits)
\end{enumerate}
\paragraph{}

Centers for Medicare and Medicaid Services (CMS) is continually working on enhancing existing policy and performances on the readmission reduction plan through accountability, and, improving processes, quality of care, and the implementation of innovative new technology. The purpose of the resource request/ improvement plan using GIS technology for hospital readmissions is to analyze "place" as a limitation to health in some cases and identify strengths of existing systems to be maintained and built upon, identify potential areas for further improvement, and support development of corrective actions.

\section{Test  Environment}
The major strengths of Loma Linda University Medical Center (LLUMC) as the test environment were identified are as follows: They were ranked No. 1 for best hospital in the Inland Empire, which operates the region's only level 1 trauma center and the only children's hospital (U.S. News & World Report). With nearly, 900 beds, each year, more than 33,000 patients are admitted and roughly half a million are served through outpatients services and over 58,000 patients are evaluated by LLUMC emergency department. In addition, Loma Linda provides assistance in governance, accounting, and fund-raising for 32 other Adventist Health hospitals and 52 clinics in 13 countries. Loma Linda University also recognize the importance of "place matters" when it comes to one's health and creating GIS applications that are integrated with Electronic Health Records (EHRs) to help improve health and quality of life and policy and decision making. Loma Linda University Medical Center continues to seek ways to improve patient care and outcomes and a GIS enabled platform will support LLUMC executive leadership and unprecedented capabilities to identify patients in real-time that are admitted and readmitted within 30 days and enhance assessment capabilities and strategies to reduce 30-day readmission rates. This adoption makes LLUMC an ideal candidate to implement the hospital readmissions dashboard as a  prototype and evaluate the robustness of the improvement plan for each objective area and condition category groups (i.e. MI, HF, PN, COPD and THA/TKA).

\section{Primary Initiatives}

The primary aim of developing the improvement plan and research is guided by Affordable Care Act policies and to help improve patient care by reducing 30-Day readmission rates and the urgency to reduce readmissions cost to Medicare of \$15 billion each year. This initiative is also aligned with Loma Linda University Medical Center, integration of Electronic Health Records (EHRs) and hospital readmissions reduction efforts on the following conditions:

\begin{itemize}

\item	Acute Myocardial Infarction (AMI) 
\item Heart Failure (HF)  
\item Pneumonia (PN)
\item Chronic Obstructive Pulmonary Disease (COPD)
\item Elective hip and knee replacements (THA/TKA)
\end{itemize}

The  research question surrounding the development of a readmissions dashboard is not based on replacing current systems but rather the question of the integration of Electronic Health Records (EHRs) and "how can a readmissions dashboard help hospitals comply with ACA policies, improve patient care and avoid preventable reimbursement penalties?" This will involve a visualization plan and rigorous evaluation of capabilities, including real-time data feeds for all hospital admissions and readmissions for the six conditions (i.e. AMI, HF, PN, COPD, THA/TKA) and routing for transitional care nurses that follow-up with patients after they are discharged. Patients at-risk for hospital readmission also benefit from the implementation plan that is patient focused because nurses and case managers will have the ability to addresses concerns in regards to patient follow-up appointments and patient comprehension of discharge and medication instructions.
\paragraph{}
The implementation plan will have several development and evaluation phases to assess the intended purpose for the application and system integration factors. Phase one includes two main functionalities. The first is by establishing a real-time linkage for spatially enabled hospital admissions data feeds of selected ICD and DRG codes. This will enable the dashboard to identify risk patients with an index admission and readmissions to the hospital and project details about their condition, vitals and demographics. The second functionality of phase one is to include routing capabilities for nurses and case managers to follow-up on patients based on readmission risks and clinical need after discharge from the hospital. 
\paragraph{}
In addition, to the improvement plan, phase 2 will enhance capabilities for a more robust platform to improve patient care at arrival and discharge from the hospital. This will also provide further capabilities for other questions regarding "what happens to patients prior to admission, during hospitalization and after discharge? " The platform will require further development of a fully integrated JavaScript application with linkages to community overlays by census tract, service areas, political boundaries, zip code and other assets, such as support groups, pharmacies and other lifestyle resources. The final phase of development will include building on the lessons learned through usability analysis from selected test environments and recommended settings for integrating Electronic Health Records (EHRs), hospital index admission and readmission measures, including transitions after discharge from the hospital.

\section{Primary InitiativesBUSINESS OVERVIEW}
Costs of Readmissions
\paragraph{}
The AHRQ summary breakdown for 30-day readmissions by payer is as follows:

\begin{itemize}

\item	Medicare (annually):				\$17 billion
\item	Private Insurances (n=600,000): 		\$8.1 billion
\item	Hospitals (n=7000,000): 			\$7.6 billion
\item	Uninsured (200,000): 	                  		\$1.5 billion
\item		Heart Failure:	                   			\$1.35 million
\item		Pneumonia:    					\$1.1 billion
\item	Hospitals affected 2013 (2,225/5,700):	\$227 million
\item		Loma Linda University MC .24\%: 		\$25 million
\begin{itemize}
Primary Areas for Improvement


\begin{figure}[ht!]
    \centering
		  % trim the pic and resize it trim: Left, Bottom, Right
    \includegraphics[trim = 40mm 1mm 2mm 10mm, clip, width=15cm, height=10cm]{C:/Users/E551910/Downloads/JamesFile/JamesFile/graph1.jpg}
    \caption{Number of Days from Discharge to Readmission for All Discharge Setting}
    \label{fig:FirstGraph}
\end{figure}

\newpage

\section{Needs Assessment}
Costs of Readmissions
\paragraph{}

% latex table generated in R 3.1.3 by xtable 1.7-4 package
% Tue Apr 28 15:54:38 2015
\begin{sidewaystable}[ht]
\centering
\begin{tabular}{lp{3cm}p{3cm}p{4cm}rlrrrp{5cm}}
  \hline
 & Component Information Systems & Device or Service & Application & Cost & Users & GIS & Data & Score & Assumptions Notes Terms \\ 
  \hline
1 & Microsoft Windows 7 & Operating system & Enterprise Base system. IT, DBA, Systems Admin & 500 &  &  &  10 &  10 & Information products required by department \\ 
  2 & Microsoft Word & Word processing & Need to know by Depart & 100 &  &  &   9 &   9 & Information products required by department \\ 
  3 & Microsoft Access, Excel & Database, Spreadsheet & Warehouse, Manage, merge analyze data and reports & 150 &  &  &   8 &   8 & Data handling, create and manage database, analysis of mandates and responsibilities \\ 
  4 & Internet & API Web Map & Email, Mesh map, web svs. &  &  &  &   9 &   9 & Information products required by department \\ 
  5 & GIS  ED Dispatch/24-hour Call center & ArcView 10 & Topographic map, Attribute info, for Emergency Response, IT, DBA, Systems Admin & 2000 &  &  10 &   7 &  17 & Historical data analysis, project best routes to take via location allocation and market share analysis and planning \\ 
  6 & Oracle/SQL  & SDE & Enterprise file gdb, IT, DBA, Systems Admin &  &  &   9 &   7 &  16 & Greater application needs and required to run SDE ArcGIS software \\ 
  7 & Tablets & Case management  services & File transfer, Field notes, IT, DBA, Systems Admin & 1800 &  &   8 &   5 &  13 &  \\ 
   \hline
\end{tabular}
\caption{Software and System Implementation}
    \label{tbl:FirstTable}
\end{sidewaystable}


\end{document}
